% Options for packages loaded elsewhere
\PassOptionsToPackage{unicode}{hyperref}
\PassOptionsToPackage{hyphens}{url}
%
\documentclass[
]{book}
\usepackage{amsmath,amssymb}
\usepackage{lmodern}
\usepackage{ifxetex,ifluatex}
\ifnum 0\ifxetex 1\fi\ifluatex 1\fi=0 % if pdftex
  \usepackage[T1]{fontenc}
  \usepackage[utf8]{inputenc}
  \usepackage{textcomp} % provide euro and other symbols
\else % if luatex or xetex
  \usepackage{unicode-math}
  \defaultfontfeatures{Scale=MatchLowercase}
  \defaultfontfeatures[\rmfamily]{Ligatures=TeX,Scale=1}
\fi
% Use upquote if available, for straight quotes in verbatim environments
\IfFileExists{upquote.sty}{\usepackage{upquote}}{}
\IfFileExists{microtype.sty}{% use microtype if available
  \usepackage[]{microtype}
  \UseMicrotypeSet[protrusion]{basicmath} % disable protrusion for tt fonts
}{}
\makeatletter
\@ifundefined{KOMAClassName}{% if non-KOMA class
  \IfFileExists{parskip.sty}{%
    \usepackage{parskip}
  }{% else
    \setlength{\parindent}{0pt}
    \setlength{\parskip}{6pt plus 2pt minus 1pt}}
}{% if KOMA class
  \KOMAoptions{parskip=half}}
\makeatother
\usepackage{xcolor}
\IfFileExists{xurl.sty}{\usepackage{xurl}}{} % add URL line breaks if available
\IfFileExists{bookmark.sty}{\usepackage{bookmark}}{\usepackage{hyperref}}
\hypersetup{
  pdftitle={A Handbook on R},
  pdfauthor={Freddy Ray Drennan},
  hidelinks,
  pdfcreator={LaTeX via pandoc}}
\urlstyle{same} % disable monospaced font for URLs
\usepackage{color}
\usepackage{fancyvrb}
\newcommand{\VerbBar}{|}
\newcommand{\VERB}{\Verb[commandchars=\\\{\}]}
\DefineVerbatimEnvironment{Highlighting}{Verbatim}{commandchars=\\\{\}}
% Add ',fontsize=\small' for more characters per line
\usepackage{framed}
\definecolor{shadecolor}{RGB}{248,248,248}
\newenvironment{Shaded}{\begin{snugshade}}{\end{snugshade}}
\newcommand{\AlertTok}[1]{\textcolor[rgb]{0.94,0.16,0.16}{#1}}
\newcommand{\AnnotationTok}[1]{\textcolor[rgb]{0.56,0.35,0.01}{\textbf{\textit{#1}}}}
\newcommand{\AttributeTok}[1]{\textcolor[rgb]{0.77,0.63,0.00}{#1}}
\newcommand{\BaseNTok}[1]{\textcolor[rgb]{0.00,0.00,0.81}{#1}}
\newcommand{\BuiltInTok}[1]{#1}
\newcommand{\CharTok}[1]{\textcolor[rgb]{0.31,0.60,0.02}{#1}}
\newcommand{\CommentTok}[1]{\textcolor[rgb]{0.56,0.35,0.01}{\textit{#1}}}
\newcommand{\CommentVarTok}[1]{\textcolor[rgb]{0.56,0.35,0.01}{\textbf{\textit{#1}}}}
\newcommand{\ConstantTok}[1]{\textcolor[rgb]{0.00,0.00,0.00}{#1}}
\newcommand{\ControlFlowTok}[1]{\textcolor[rgb]{0.13,0.29,0.53}{\textbf{#1}}}
\newcommand{\DataTypeTok}[1]{\textcolor[rgb]{0.13,0.29,0.53}{#1}}
\newcommand{\DecValTok}[1]{\textcolor[rgb]{0.00,0.00,0.81}{#1}}
\newcommand{\DocumentationTok}[1]{\textcolor[rgb]{0.56,0.35,0.01}{\textbf{\textit{#1}}}}
\newcommand{\ErrorTok}[1]{\textcolor[rgb]{0.64,0.00,0.00}{\textbf{#1}}}
\newcommand{\ExtensionTok}[1]{#1}
\newcommand{\FloatTok}[1]{\textcolor[rgb]{0.00,0.00,0.81}{#1}}
\newcommand{\FunctionTok}[1]{\textcolor[rgb]{0.00,0.00,0.00}{#1}}
\newcommand{\ImportTok}[1]{#1}
\newcommand{\InformationTok}[1]{\textcolor[rgb]{0.56,0.35,0.01}{\textbf{\textit{#1}}}}
\newcommand{\KeywordTok}[1]{\textcolor[rgb]{0.13,0.29,0.53}{\textbf{#1}}}
\newcommand{\NormalTok}[1]{#1}
\newcommand{\OperatorTok}[1]{\textcolor[rgb]{0.81,0.36,0.00}{\textbf{#1}}}
\newcommand{\OtherTok}[1]{\textcolor[rgb]{0.56,0.35,0.01}{#1}}
\newcommand{\PreprocessorTok}[1]{\textcolor[rgb]{0.56,0.35,0.01}{\textit{#1}}}
\newcommand{\RegionMarkerTok}[1]{#1}
\newcommand{\SpecialCharTok}[1]{\textcolor[rgb]{0.00,0.00,0.00}{#1}}
\newcommand{\SpecialStringTok}[1]{\textcolor[rgb]{0.31,0.60,0.02}{#1}}
\newcommand{\StringTok}[1]{\textcolor[rgb]{0.31,0.60,0.02}{#1}}
\newcommand{\VariableTok}[1]{\textcolor[rgb]{0.00,0.00,0.00}{#1}}
\newcommand{\VerbatimStringTok}[1]{\textcolor[rgb]{0.31,0.60,0.02}{#1}}
\newcommand{\WarningTok}[1]{\textcolor[rgb]{0.56,0.35,0.01}{\textbf{\textit{#1}}}}
\usepackage{longtable,booktabs,array}
\usepackage{calc} % for calculating minipage widths
% Correct order of tables after \paragraph or \subparagraph
\usepackage{etoolbox}
\makeatletter
\patchcmd\longtable{\par}{\if@noskipsec\mbox{}\fi\par}{}{}
\makeatother
% Allow footnotes in longtable head/foot
\IfFileExists{footnotehyper.sty}{\usepackage{footnotehyper}}{\usepackage{footnote}}
\makesavenoteenv{longtable}
\usepackage{graphicx}
\makeatletter
\def\maxwidth{\ifdim\Gin@nat@width>\linewidth\linewidth\else\Gin@nat@width\fi}
\def\maxheight{\ifdim\Gin@nat@height>\textheight\textheight\else\Gin@nat@height\fi}
\makeatother
% Scale images if necessary, so that they will not overflow the page
% margins by default, and it is still possible to overwrite the defaults
% using explicit options in \includegraphics[width, height, ...]{}
\setkeys{Gin}{width=\maxwidth,height=\maxheight,keepaspectratio}
% Set default figure placement to htbp
\makeatletter
\def\fps@figure{htbp}
\makeatother
\setlength{\emergencystretch}{3em} % prevent overfull lines
\providecommand{\tightlist}{%
  \setlength{\itemsep}{0pt}\setlength{\parskip}{0pt}}
\setcounter{secnumdepth}{5}
\usepackage{booktabs}
\ifluatex
  \usepackage{selnolig}  % disable illegal ligatures
\fi
\usepackage[]{natbib}
\bibliographystyle{apalike}

\title{A Handbook on R}
\author{Freddy Ray Drennan}
\date{2021-11-22}

\begin{document}
\maketitle

{
\setcounter{tocdepth}{1}
\tableofcontents
}
\hypertarget{introduction}{%
\chapter{Introduction}\label{introduction}}

\hypertarget{a-dose-of-r}{%
\section{A Dose of R}\label{a-dose-of-r}}

Let's solve a problem using R. Suppose we have a friend that is interested in the current trend regarding COVID-19 cases. The first thing we will probably do is try to figure out an efficient and reliable way for importing Covid-19 data into our R session. Conveniently, the \href{https://covid19datahub.io/articles/r.html}{\texttt{COVID19}} package allows us to pull the latest data without any hard work and consists of one function - \texttt{covid19}.

A function does stuff given a set of inputs. Remember seeing equations like \(y = 2x + 3\) in algebra? We said given \(x=2\) then \(y=7\). because \(2(2) + 3 = 7\). In calculus, equations became functions with a slightly different syntax. \(f(x) = 2x + 3\). What happened to the \(y\)? Well, nothing really. We can say \(y=f(x)=x^2\) or \(result=f(x)\) or \(z=g(x,y)=x + min(x, y)\). Once we have these definitions, we can go further and compose them together. For example, \(f(g(x, y)) = (x + min(x, y))^2\). Now, the outputs of functions become the inputs for other functions.

\begin{Shaded}
\begin{Highlighting}[]
\FunctionTok{library}\NormalTok{(cli)}

\CommentTok{\# Same as f(x) = 2x + 3}
\NormalTok{f }\OtherTok{\textless{}{-}}  \ControlFlowTok{function}\NormalTok{(x) \{}
\NormalTok{  x }\OtherTok{\textless{}{-}}\NormalTok{ x }\SpecialCharTok{*}\NormalTok{ x}
\NormalTok{  x}
\NormalTok{\}}

\NormalTok{g }\OtherTok{\textless{}{-}} \ControlFlowTok{function}\NormalTok{(}\AttributeTok{x=}\ConstantTok{NULL}\NormalTok{, }\AttributeTok{y=}\ConstantTok{NULL}\NormalTok{) \{}
\NormalTok{  result }\OtherTok{\textless{}{-}}\NormalTok{ x }\SpecialCharTok{+} \FunctionTok{min}\NormalTok{(x, y)}
\NormalTok{  result}
\NormalTok{\}}

\FunctionTok{print}\NormalTok{(}\FunctionTok{f}\NormalTok{(}\FunctionTok{g}\NormalTok{(}\DecValTok{3}\NormalTok{, }\DecValTok{4}\NormalTok{)))}
\end{Highlighting}
\end{Shaded}

\begin{verbatim}
## [1] 36
\end{verbatim}

We now have a way of describing inputs and output a little more clearly. Instead of writing, \((3 + min(3, 4)) * (3 + min(3, 4))\) we can write \(f(g(3, 4))\) or try new creations like \(z(x, y)=f(g(f(x), f(y)))\) so \(z(1, 2)=f(g(f(1), f(2)))=f(g(1, 4))=f(2)=4\).\\
~\\
Now just take this idea about functions and expand your definition of inputs and outputs to be any number, none or many, and of any type that R supports - character, numeric, date/time, data.frame or list - all of which we'll cover.\\

In your RStudio console, you can write the following to install the \texttt{COVID19} package using the \texttt{install.packages} function. If you are interested in learning more about this function, you can write \texttt{?install.packages} in your console and the documentation for the function will appear.

\begin{Shaded}
\begin{Highlighting}[]
\CommentTok{\# For help menu, uncomment next line}
\CommentTok{\# ?install.packages }

\CommentTok{\# If the package is not yet installed, you can install it by passing }
\CommentTok{\# a string with the package name to the \textasciigrave{}install.packages\textasciigrave{} function}
\FunctionTok{install.packages}\NormalTok{(}\AttributeTok{pkgs =} \FunctionTok{c}\NormalTok{(}\StringTok{\textquotesingle{}COVID19\textquotesingle{}}\NormalTok{))}
\end{Highlighting}
\end{Shaded}

For a full list of what packages are available through the \texttt{install.packages} function, please check out the \href{https://cran.r-project.org/web/packages/index.html}{Contributed Packages} page at CRAN.

\begin{itemize}
\tightlist
\item
  \emph{WRITE ABOUT HELP MENU?}
\end{itemize}

I'm a man of few words. So let's get to it. The two lines of code below consist of three functions. The first two are \texttt{library} and \texttt{covid19}, but the third is hidden, and I will disclose where shortly. \emph{Functions} are spaces for stuff to happen. Functions help us make common procedures repeatable. By creating a function with a particular name and inputs, we can get some sort of useful (or not useful, the world's your oyster) output.

In this case, \texttt{library} loads \emph{packages} from a folder in the R environment called library. You can see which ones your R environment knows about by running the function \texttt{.libPaths()}. \texttt{covid19} is a function from the \texttt{COVID19} package, and would only be available after executing \texttt{library(COVID19)} or if \texttt{library(COVID19)} is omitted, by pulling it from the package namespace directly by preceding the function with the package name and two colons like so: \texttt{COVID19::covid19}. Generally speaking, you simply use \texttt{library} because it reduces the amount of text on the page.

\begin{Shaded}
\begin{Highlighting}[]
\FunctionTok{library}\NormalTok{(COVID19)}
\end{Highlighting}
\end{Shaded}

\begin{verbatim}
## Warning: package 'COVID19' was built under R version 4.1.2
\end{verbatim}

\begin{Shaded}
\begin{Highlighting}[]
\NormalTok{covid\_data }\OtherTok{\textless{}{-}} \FunctionTok{covid19}\NormalTok{(}
    \AttributeTok{country =} \StringTok{\textquotesingle{}United States\textquotesingle{}}\NormalTok{, }
    \AttributeTok{start =} \StringTok{\textquotesingle{}2021{-}01{-}01\textquotesingle{}}\NormalTok{, }
    \AttributeTok{end =} \StringTok{"2021{-}11{-}21"}\NormalTok{,}
    \AttributeTok{verbose =} \ConstantTok{FALSE}
\NormalTok{)}
\end{Highlighting}
\end{Shaded}

\href{https://stackoverflow.com/questions/58568392/how-do-i-know-a-function-or-an-operation-in-r-is-vectorized\#:~:text=To\%20identify\%20if\%20an\%20R\%20object\%20is\%20a\%20vector\%20\%2C\%20I,a\%20vector\%20or\%20False\%20otherwise.}{How do I know if a function is vectorized}

\hypertarget{initial-setup}{%
\section{Initial Setup}\label{initial-setup}}

\href{https://hackmd.io/vGRGEPo8QQyiG8gecWv71g}{Book Outline}

\begin{itemize}
\item
  \href{https://cran.r-project.org/}{Install R}
\item
  \href{https://www.rstudio.com/products/rstudio/download/}{Install R Studio}
\item
  \href{https://cran.r-project.org/bin/windows/Rtools/}{Windows Only: Install RTools}

  \begin{itemize}
  \tightlist
  \item
    When installed, run in the RStudio Console: \texttt{write(\textquotesingle{}PATH="\$\{RTOOLS40\_HOME\}\textbackslash{}\textbackslash{}usr\textbackslash{}\textbackslash{}bin;\$\{PATH\}"\textquotesingle{},\ file\ =\ "\textasciitilde{}/.Renviron",\ append\ =\ TRUE)}
  \end{itemize}
\item
  \href{https://www.omgubuntu.co.uk/how-to-install-wsl2-on-windows-10}{Windows Only: Install WSL2}

  \begin{itemize}
  \tightlist
  \item
    Computer should be completely updated before install.
  \end{itemize}
\item
  \href{https://git-scm.com/downloads}{Install Git}
\item
  \href{https://github.com/}{Create Github Account}
\item
  \href{https://github.com/fdrennan/r-handbook}{Fork r-handbook}
\item
  \href{https://docs.docker.com/get-docker/}{Install Docker and Docker Compose}
\item
  \href{https://aws.amazon.com/}{Create AWS Account}

  \begin{itemize}
  \tightlist
  \item
    Billing will be discussed in the course, but don't expect to pay much - i.e., 10-20 dollars a month for high course activity.
  \item
    Remember to \texttt{stop} EC2 servers when we begin using them. AWS is polite about your first few refund requests.
  \end{itemize}
\item
  \href{reddit.com}{Create Reddit Account}

  \begin{itemize}
  \tightlist
  \item
    \href{https://towardsdatascience.com/how-to-use-the-reddit-api-in-python-5e05ddfd1e5c}{Follow Instructions here}
  \end{itemize}
\end{itemize}

Make sure you install the \href{https://www.tidyverse.org/}{\texttt{tidyverse}} packages. Update to renv later.

\begin{Shaded}
\begin{Highlighting}[]
\FunctionTok{install.packages}\NormalTok{(}\StringTok{\textquotesingle{}tidyverse\textquotesingle{}}\NormalTok{)}
\end{Highlighting}
\end{Shaded}

\hypertarget{what-is-r}{%
\chapter{What is R}\label{what-is-r}}

\hypertarget{types-of-problems-you-can-solve}{%
\section{Types of Problems You Can Solve}\label{types-of-problems-you-can-solve}}

\hypertarget{base-r-tidyverse-data.table}{%
\section{Base R, Tidyverse, data.table}\label{base-r-tidyverse-data.table}}

\hypertarget{arguments-developments-within-the-language}{%
\section{Arguments/ Developments within the language}\label{arguments-developments-within-the-language}}

\hypertarget{what-are-variables}{%
\section{What are Variables}\label{what-are-variables}}

\hypertarget{valid-variable-names}{%
\subsection{Valid Variable Names}\label{valid-variable-names}}

\hypertarget{functions}{%
\chapter{Building Blocks}\label{functions}}

\hypertarget{vectors}{%
\section{Vectors}\label{vectors}}

Vectors are containers information of similar type. You can think of them as having \(1*n\) cells where \(n\) is any positive integer, and make up the rows and columns of tables. Vectors always contain the same type of value. R has many different types of vectors, but the most common are \textbf{numeric}, \textbf{character}, and \textbf{logical} (\textbf{TRUE}/\textbf{FALSE}).

Vectors are cool. I like to think of them as boxes that can only be stacked on top of one another.

\begin{Shaded}
\begin{Highlighting}[]
\FunctionTok{typeof}\NormalTok{(}\FunctionTok{c}\NormalTok{(}\ConstantTok{TRUE}\NormalTok{))}
\end{Highlighting}
\end{Shaded}

\begin{verbatim}
## [1] "logical"
\end{verbatim}

\begin{Shaded}
\begin{Highlighting}[]
\FunctionTok{typeof}\NormalTok{(}\FunctionTok{c}\NormalTok{(}\ConstantTok{TRUE}\NormalTok{, }\DecValTok{1}\NormalTok{))}
\end{Highlighting}
\end{Shaded}

\begin{verbatim}
## [1] "double"
\end{verbatim}

\begin{Shaded}
\begin{Highlighting}[]
\FunctionTok{typeof}\NormalTok{(}\FunctionTok{c}\NormalTok{(}\ConstantTok{TRUE}\NormalTok{, }\DecValTok{1}\NormalTok{, }\StringTok{\textquotesingle{}a\textquotesingle{}}\NormalTok{))}
\end{Highlighting}
\end{Shaded}

\begin{verbatim}
## [1] "character"
\end{verbatim}

\hypertarget{functions-1}{%
\section{Functions}\label{functions-1}}

\textbf{Functions} are containers where anything or nothing can happen, but whatever happens, it happens the same way every single time. They allow for generalization of complicated ideas and routines that we wish to repeat over and over again. A function may have an input, but no output. It may have an output, but no input, both or none. If it's something you need to do repeatedly, or containing code makes your program easier to read, then write a function for that process.

\textbf{Rule 4: Functions have inputs, outputs, and a body.} A function can have multiple outputs, but given a particular set of inputs, the solution should never change assuming you are not developing a function with randomness built in.

R has a built-in \href{https://stat.ethz.ch/R-manual/R-devel/library/base/html/Constants.html}{constant} called \texttt{letters}. This means that no matter where you are writing R, \texttt{letters} will be available to you. We see that \texttt{letters} is a \textbf{character} \textbf{vector} in our program below, and use the composition of functions to create a program that describes \texttt{letters}.

\begin{Shaded}
\begin{Highlighting}[]
\FunctionTok{print}\NormalTok{(letters)}
\end{Highlighting}
\end{Shaded}

\begin{verbatim}
##  [1] "a" "b" "c" "d" "e" "f" "g" "h" "i" "j" "k" "l" "m" "n" "o" "p" "q" "r" "s"
## [20] "t" "u" "v" "w" "x" "y" "z"
\end{verbatim}

Next, we can use some functions which take in pretty much any object that exists in R and spits back information regarding the \texttt{letters} data.

\begin{Shaded}
\begin{Highlighting}[]
\NormalTok{main }\OtherTok{\textless{}{-}} \ControlFlowTok{function}\NormalTok{() \{}
\NormalTok{  print\_information }\OtherTok{\textless{}{-}} \ControlFlowTok{function}\NormalTok{(x) \{}
    
\NormalTok{    variable\_name }\OtherTok{=} \FunctionTok{deparse1}\NormalTok{(}\FunctionTok{substitute}\NormalTok{(x))}
    
\NormalTok{    length\_x }\OtherTok{=} \FunctionTok{length}\NormalTok{(x)}
\NormalTok{    typeof\_x }\OtherTok{\textless{}{-}} \FunctionTok{typeof}\NormalTok{(x)}
\NormalTok{    is\_vec\_x }\OtherTok{\textless{}{-}} \FunctionTok{is.vector}\NormalTok{(x)}
    
\NormalTok{    meta\_list }\OtherTok{\textless{}{-}} \FunctionTok{list}\NormalTok{(}
      \AttributeTok{length =}\NormalTok{ length\_x, }
      \AttributeTok{type =}\NormalTok{ typeof\_x, }
      \AttributeTok{is\_vector =}\NormalTok{ is\_vec\_x}
\NormalTok{    )}
    
\NormalTok{    cli}\SpecialCharTok{::}\FunctionTok{cli\_alert}\NormalTok{(}\StringTok{\textquotesingle{}Information about \{variable\_name\}\textquotesingle{}}\NormalTok{)}
    
\NormalTok{    cli}\SpecialCharTok{::}\FunctionTok{cli\_alert\_info}\NormalTok{(}\StringTok{"\{variable\_name\} is a 1x\{length\_x\} dimensional"}\NormalTok{)}
\NormalTok{    cli}\SpecialCharTok{::}\FunctionTok{cli\_alert\_info}\NormalTok{(}\StringTok{""}\NormalTok{)}
    
\NormalTok{    purrr}\SpecialCharTok{::}\FunctionTok{iwalk}\NormalTok{(meta\_list, }\ControlFlowTok{function}\NormalTok{(x, index) \{}
\NormalTok{      cli}\SpecialCharTok{::}\FunctionTok{cli\_alert\_info}\NormalTok{(glue}\SpecialCharTok{::}\FunctionTok{glue}\NormalTok{(}\StringTok{\textquotesingle{}\{index\} \{x\} is type \{typeof(x)\}\textquotesingle{}}\NormalTok{))}
\NormalTok{    \})}
    
    \FunctionTok{return}\NormalTok{(meta\_list)}
\NormalTok{  \}}
  
\NormalTok{  cli}\SpecialCharTok{::}\FunctionTok{cli\_alert\_info}\NormalTok{(}\StringTok{\textquotesingle{}Execute print\_information\textquotesingle{}}\NormalTok{)}
\NormalTok{  output }\OtherTok{\textless{}{-}} \FunctionTok{print\_information}\NormalTok{(mtcars)}
\NormalTok{  cli}\SpecialCharTok{::}\FunctionTok{cli\_alert\_success}\NormalTok{(}\StringTok{\textquotesingle{}Execute print\_information complete\textquotesingle{}}\NormalTok{)}
  
  \FunctionTok{print}\NormalTok{(output)}
\NormalTok{\}}

\FunctionTok{main}\NormalTok{()}
\end{Highlighting}
\end{Shaded}

\begin{verbatim}
## i Execute print_information
\end{verbatim}

\begin{verbatim}
## > Information about mtcars
\end{verbatim}

\begin{verbatim}
## i mtcars is a 1x11 dimensional
\end{verbatim}

\begin{verbatim}
## i
\end{verbatim}

\begin{verbatim}
## i length 11 is type integer
\end{verbatim}

\begin{verbatim}
## i type list is type character
\end{verbatim}

\begin{verbatim}
## i is_vector FALSE is type logical
\end{verbatim}

\begin{verbatim}
## v Execute print_information complete
\end{verbatim}

\begin{verbatim}
## $length
## [1] 11
## 
## $type
## [1] "list"
## 
## $is_vector
## [1] FALSE
\end{verbatim}

\hypertarget{debugging}{%
\chapter{Debugging}\label{debugging}}

\hypertarget{what-is-the-debugger}{%
\section{What is the debugger?}\label{what-is-the-debugger}}

\hypertarget{how-to-learn-r-without-knowing-any-r}{%
\section{How to learn R without knowing any R}\label{how-to-learn-r-without-knowing-any-r}}

\hypertarget{browser}{%
\section{\texorpdfstring{\texttt{browser()}}{browser()}}\label{browser}}

\hypertarget{next-continue}{%
\section{\texorpdfstring{\texttt{next}, \texttt{continue}}{next, continue}}\label{next-continue}}

\hypertarget{debug-and-undebug}{%
\section{\texorpdfstring{\texttt{debug} and \texttt{undebug}}{debug and undebug}}\label{debug-and-undebug}}

\hypertarget{debugonce}{%
\section{\texorpdfstring{\texttt{debugonce}}{debugonce}}\label{debugonce}}

\hypertarget{understanding-debugging-output}{%
\section{Understanding debugging output}\label{understanding-debugging-output}}

\hypertarget{lots-of-debugging-exercises-cannot-stress-enough}{%
\section{LOTS OF DEBUGGING EXERCISES CANNOT STRESS ENOUGH}\label{lots-of-debugging-exercises-cannot-stress-enough}}

\hypertarget{vectors-1}{%
\chapter{Vectors}\label{vectors-1}}

\hypertarget{c}{%
\section{\texorpdfstring{\texttt{c}}{c}}\label{c}}

\hypertarget{and}{%
\section{\texorpdfstring{\texttt{{[}} and \texttt{{[}{[}}}{{[} and {[}{[}}}\label{and}}

\begin{itemize}
\tightlist
\item
  Vectors

  \begin{itemize}
  \tightlist
  \item
    atomic
  \end{itemize}
\item
  Strings

  \begin{itemize}
  \tightlist
  \item
    Base R
  \item
    \texttt{stringr}

    \begin{itemize}
    \tightlist
    \item
      Regular Expressions
    \end{itemize}
  \item
    \href{https://raw.githubusercontent.com/rstudio/cheatsheets/main/strings.pdf}{Cheat Sheet}
  \end{itemize}
\item
  Numbers

  \begin{itemize}
  \tightlist
  \item
    Integer
  \item
    Double
  \end{itemize}
\item
  Factors

  \begin{itemize}
  \tightlist
  \item
    \texttt{as.factor} vs.~\texttt{as\_factor}
  \end{itemize}
\item
  Dates

  \begin{itemize}
  \tightlist
  \item
    Base R
  \item
    \texttt{lubridate}
  \end{itemize}
\end{itemize}

\hypertarget{lists}{%
\chapter{Lists}\label{lists}}

\hypertarget{list}{%
\section{\texorpdfstring{\texttt{list}}{list}}\label{list}}

\hypertarget{and-1}{%
\section{\texorpdfstring{\texttt{{[}} and \texttt{{[}{[}}}{{[} and {[}{[}}}\label{and-1}}

\begin{itemize}
\tightlist
\item
  Lists

  \begin{itemize}
  \tightlist
  \item
    \texttt{list()} and \texttt{c}
  \item
    \texttt{{[}} and \texttt{{[}{[}}
  \item
    Connection between lists and json

    \begin{itemize}
    \tightlist
    \item
      \texttt{jsonlite}
    \end{itemize}
  \end{itemize}
\end{itemize}

\hypertarget{tables}{%
\chapter{Tables}\label{tables}}

\hypertarget{c-1}{%
\section{\texorpdfstring{\texttt{c}}{c}}\label{c-1}}

\hypertarget{and-2}{%
\section{\texorpdfstring{\texttt{{[}} and \texttt{{[}{[}}}{{[} and {[}{[}}}\label{and-2}}

\begin{itemize}
\tightlist
\item
  Tables

  \begin{itemize}
  \tightlist
  \item
    matrices
  \item
    \texttt{data.frame} vs \texttt{tibble}
  \item
    data.frames are lists with equal length, atomic vectors
  \end{itemize}
\end{itemize}

\hypertarget{functional-programming}{%
\chapter{Functional Programming}\label{functional-programming}}

\begin{enumerate}
\def\labelenumi{\arabic{enumi}.}
\setcounter{enumi}{1}
\tightlist
\item
  \href{https://hackmd.io/R_cVl-DBSve03vc1trHrGw}{Functions}

  \begin{itemize}
  \tightlist
  \item
    Sequences
  \item
    Mapping functions
  \item
    pipes
  \item
    void
  \item
    \texttt{return}

    \begin{itemize}
    \tightlist
    \item
      Can a function return nothing?
    \item
      What are side effects?
    \item
      Multiple return statements
    \end{itemize}
  \end{itemize}
\end{enumerate}

\hypertarget{base-r}{%
\section{Base R}\label{base-r}}

\texttt{apply}, \texttt{lapply}, \texttt{mapply}

\hypertarget{modern-r}{%
\section{Modern R}\label{modern-r}}

\href{https://raw.githubusercontent.com/rstudio/cheatsheets/main/purrr.pdf}{\texttt{purrr}}
* \texttt{map\_*}
* \texttt{map2\_*}
* \texttt{pmap\_*}
* Iterate over What?
* Why are data.frames mapped over columnwise?
* A: data.frames are lists, and mapping functions will iterate over each individual item in a list

\hypertarget{tidy-data}{%
\chapter{Tidy Data}\label{tidy-data}}

\begin{itemize}
\tightlist
\item
  Concept of tidy data

  \begin{itemize}
  \tightlist
  \item
    \href{https://vita.had.co.nz/papers/tidy-data.pdf}{Tidy Data Paper}
  \end{itemize}
\item
  \texttt{tidyr}

  \begin{itemize}
  \tightlist
  \item
    \texttt{pivot\_longer}
  \item
    \texttt{pivot\_wider}
  \end{itemize}
\end{itemize}

\hypertarget{dplyr}{%
\chapter{dplyr}\label{dplyr}}

\begin{itemize}
\tightlist
\item
  \texttt{dplyr} and data manipulation

  \begin{itemize}
  \tightlist
  \item
    main functions

    \begin{itemize}
    \tightlist
    \item
      \texttt{select}
    \item
      \texttt{mutate}
    \item
      \texttt{filter}
    \item
      \texttt{transmute}
    \end{itemize}
  \item
    summarizing data

    \begin{itemize}
    \tightlist
    \item
      \texttt{group\_by}
    \item
      \texttt{summarize} - one row per group
    \item
      \texttt{mutate} - one or many rows per group will have same value
    \item
      \texttt{ungroup} - remove grouping

      \begin{itemize}
      \tightlist
      \item
        Not everything has to be a \texttt{group\_by}
      \item
        Solving group problems with vectors
      \end{itemize}
    \end{itemize}
  \end{itemize}
\end{itemize}

\begin{itemize}
\tightlist
\item
  Joining Tables

  \begin{itemize}
  \tightlist
  \item
    \texttt{inner\_join}
  \item
    \texttt{full\_join}
  \item
    \texttt{left\_join} / \texttt{right\_join}
  \end{itemize}
\end{itemize}

\hypertarget{project-outline}{%
\chapter{Project Outline}\label{project-outline}}

To be expanded over many chapters

\begin{enumerate}
\def\labelenumi{\arabic{enumi}.}
\tightlist
\item
  Windows vs Mac vs Linux
\item
  Docker Installation

  \begin{itemize}
  \tightlist
  \item
    Windows needs to set up VM in bios
  \end{itemize}
\item
  RStudio IDE

  \begin{itemize}
  \tightlist
  \item
    \href{https://raw.githubusercontent.com/rstudio/cheatsheets/main/rstudio-ide.pdf}{Cheat Sheet}
  \end{itemize}
\item
  reddit api creds
\item
  reticulate

  \begin{itemize}
  \tightlist
  \item
    Enough R to know Python

    \begin{itemize}
    \tightlist
    \item
      \href{https://rstudio.github.io/reticulate/}{Type Conversions}
    \end{itemize}
  \item
    miniconda installation
  \item
    virtual environments
  \end{itemize}
\item
  Package Structure

  \begin{itemize}
  \tightlist
  \item
    Defaults for RStudio

    \begin{itemize}
    \tightlist
    \item
      Rebuild and Restart with Roxygen2
    \end{itemize}
  \item
    \texttt{.env}
  \item
    \texttt{.gitignore}
  \item
    \texttt{.Rprofile}
  \item
    \texttt{.Renviron}
  \item
    Packages necessary for efficient development

    \begin{itemize}
    \tightlist
    \item
      usethis
    \item
      roxygen
    \item
      devtools

      \begin{itemize}
      \tightlist
      \item
        \href{https://raw.githubusercontent.com/rstudio/cheatsheets/main/package-development.pdf}{Cheat Sheet}
      \end{itemize}
    \end{itemize}
  \item
    Make and Makefiles

    \begin{itemize}
    \tightlist
    \item
      Automating Package Build
    \end{itemize}
  \item
    Unit Testing (probably bad location for ut, no code written)

    \begin{itemize}
    \tightlist
    \item
      testthat
    \end{itemize}
  \end{itemize}
\item
  Git

  \begin{itemize}
  \tightlist
  \item
    Github
  \item
    git circle, workflow
  \end{itemize}
\item
  Retrieving Data from API

  \begin{itemize}
  \tightlist
  \item
    \texttt{praw}
  \item
    \texttt{dotenv} and \texttt{.env}
  \item
    \href{https://github.com/fdrennan/ndexr-platform/blob/master/redditor-api/R/reddit.R}{Old Reddit Code to start with}
  \end{itemize}
\item
  Docker and Docker Compose Introduction

  \begin{itemize}
  \tightlist
  \item
    \texttt{.dockerignore}
  \end{itemize}
\item
  Create Postgres Database

  \begin{itemize}
  \tightlist
  \item
    What are Ports?
  \item
    Postgres Credentials
  \end{itemize}
\item
  Create functions for Storing Reddit Data
\item
  \emph{Need preferred method for streaming data, i.e., Airflow not a good scheduler for scripts that are always running} and need a kickstart on failure, timeout, etc. Docker with \texttt{restart:\ always} may be sufficient
\item
  Plumber API

  \begin{itemize}
  \tightlist
  \item
    Add to docker-compose
  \item
    Functions for ETL, Shiny Application
  \end{itemize}
\item
  ETL with Airflow and HTTP operator connected to Plumber API
\item
  Shiny

  \begin{itemize}
  \tightlist
  \item
    Reactive Graph

    \begin{itemize}
    \tightlist
    \item
      Order does not matter, the graph does
    \end{itemize}
  \item
    Why Modules?

    \begin{itemize}
    \tightlist
    \item
      \texttt{map} over modules
    \end{itemize}
  \end{itemize}
\item
  Automating Infrastructure

  \begin{itemize}
  \tightlist
  \item
    \texttt{awscli}
  \item
    \texttt{boto3}

    \begin{itemize}
    \tightlist
    \item
      \href{https://github.com/fdrennan/ndexr-platform/tree/master/biggr/R}{\texttt{biggr}}
    \end{itemize}
  \item
    Create EC2 Server from R

    \begin{itemize}
    \tightlist
    \item
      User Data
    \end{itemize}
  \end{itemize}
\end{enumerate}

  \bibliography{book.bib,packages.bib}

\end{document}
